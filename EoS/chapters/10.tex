
\chapter{10}

Elide Lochan stood before a creature birthed from a dark god's nightmares.

Across the clearing, it towered over her, its talons digging into the loam of the forest floor.
"There you are," it hissed through teeth sharper than a fish's.

"Come with me, girl, and I will grant you a quick end."

Lies.
She saw how it sized her up, claws curling as if it could already feel them shredding into her soft belly.
The thing had appeared in her path as if a cloud of night had dropped it there, and had laughed when she screamed.
Her knife shook as she raised it.

It stood like a man---spoke like one.
And its eyes\ldots Utterly soulless, yet the shape of them\ldots They were human, too.
Monstrous---what terrible mind had dreamed up such a thing?

She knew the answer.

Help.
She needed help.
But that man from the stream was likely dead at the claws of the other beasts.
She wondered how long that magic of his had held out.

The creature stepped toward her, its muscled legs closing the distance too quickly.
She backed toward the trees, the direction she'd come from.

"Is your blood as sweet as your face, girl?"
Its grayish tongue tasted the air between them.

\emph{Think, think, think}.

What would Manon do before such a creature?

Manon, she remembered, came equipped with claws and fangs of her own.

But a small voice whispered in her ear, \emph{So do you}.
\emph{Use what you have}.

There were other weapons than those made of iron and steel.

Though her knees shook, Elide lifted her chin and met the black, human eyes of the creature.

"Careful," she said, dropping her voice into the purr Manon had so often used to frighten the wits out of everyone.
Elide reached into the pocket of her coat, pulling out the shard of stone and clenching it in her fist, willing that otherworldly presence to fill the clearing, the world.
She prayed the creature wouldn't look at her fist, wouldn't ask what was in it as she drawled, "Do you think the Dark King will be pleased if you harm me?"
She looked down her nose at it.
Or as best as she could while standing several feet shorter.
"I have been sent to look for the girl.
Do not interfere."

The creature seemed to recognize the fighting leathers then.

Seemed to scent that strange, \emph{off} scent surrounding the rock.

And it hesitated.

Elide kept her face a mask of cold displeasure.
"Get out of my sight."

She almost vomited as she began stalking toward it, toward sure death.
But she stomped along, prowling as Manon had so often done.
Elide made herself look up into the bat-like, hideous face as she passed.
"Tell your brethren that if you interfere again, I will personally oversee what delights you experience upon Morath's tables."

Doubt still danced in its eyes---along with real fear.
A lucky guess, those words and phrases, based on what she'd overheard.
She didn't let herself consider what had been done to make such a creature quake at the mention.

Elide was five paces from the creature, keenly aware that her spine was now vulnerable to those shredding claws and teeth, when it asked, "Why did you flee at our approach?"

She said without turning, in that cold, vicious voice of Manon Blackbeak, "I do not tolerate the questions of underlings.
You have already disrupted my hunt and injured my ankle with your useless attack.
Pray that I do not remember your face when I return to the Keep."

She knew her mistake the moment it sucked in a hissing breath.

Still, she kept her legs moving, back straight.

"What a coincidence," it mused, "that our prey is similarly lamed."

Anneith save her.
Perhaps it had not noticed the limp until then.
Fool.
\emph{Fool}.

Running would do her no good---running would proclaim the creature had won, that it was right.
She halted, as if her temper had yanked on its leash, and snapped her face toward the creature.
"What is that you're hissing about?"
Utter conviction, utter rage.

Again the creature paused.
One chance---just one chance.
It'd learn soon enough that it had been duped.

Elide held its gaze.
It was like staring a dead snake in the eyes.

She said with that lethal quiet the witches liked to use, "Do not make me reveal what His Dark Majesty put inside \emph{me} on that table."

As if in response, the stone in her hand throbbed, and she could have sworn darkness flickered.

The creature shuddered, backing away a step.

Elide didn't consider what she held as she sneered one last time and stalked away.

She made it perhaps half a mile before the forest was again full of chittering life.

She fell to her knees and vomited.

Nothing but bile and water came out.
She was so busy hurling up her guts with stupid fear and relief that she didn't notice anyone's approach until it was too late.

A broad hand clamped on her shoulder, whirling her around.

She drew her dagger, but too slowly.
The same hand released her to slap the blade to the grass.

Elide found herself staring into the dirt-splattered face of the man from the stream.
No, not dirt.
Blood that reeked---black blood.

"How?"
she said, stumbling away a step.

"\emph{You first}," he snarled, but whipped his head toward the forest behind them.

She followed his gaze.
Saw nothing.

When she looked at his harsh face, a sword lay against her throat.

She tried to fall back, but he gripped her arm, holding her as steel bit into her skin.
"Why do you smell of one of them?
Why do they chase you?"

She'd pocketed the stone, or else she might have shown him.
But movement might cause him to strike---and that small voice whispered to keep the stone concealed.

She offered another truth.
"Because I have spent the past several months in Morath, living amongst that scent.
They seek me because I managed to get free.
I flee north---to safety."

Faster than she could see, he lowered his blade---only to slice it across her arm.
A scratch, barely more than a whisper of pain.

They both watched as her red blood surged and dribbled.

It seemed answer enough for him.

"You can call me Lorcan," he said, though she hadn't asked.
And with that, he hauled her over his broad shoulder like a sack of potatoes and ran.

Elide knew two things within seconds:

That the remaining creatures---however many there were---had to be on their trail and closing in fast.
Had to have realized she'd bluffed her way free.

And that the man, moving swift as a wind between the oaks, was demi-Fae.

\begin{center}
	\includegraphics[width=0.65in,height=0.13in]{images/seperator}
\end{center}

Lorcan ran and ran, his lungs gobbling down great gulps of the forest's stifling air.
Slung over his shoulder, the girl didn't even whimper as the miles passed.
He'd carried packs heavier than her over entire mountain ranges.

Lorcan slowed when his strength at last began to flag, spent quicker thanks to the magic he'd used to get those three beasts into a stranglehold, battering past their natural-born immunity to it, then kill two while he pinned the other long enough to sprint for the girl.

He'd been lucky.

The girl, it seemed, had been smart.

He jogged into a stop, setting her down hard enough that she winced--- winced and hopped a bit on that hurt ankle.
Her blood had flowed red instead of the reeking black that implied Valg possession, but it still didn't explain how she'd been able to intimidate that ilken into submission.

"Where are we going?"
she said, swinging her pack to pull out her canteen.
He waited for the tears and prayers and begging.
She just unscrewed the cap of the leather-coated container and swigged deep.
Then, to his surprise, offered him some.

Lorcan didn't take it.
She merely drank again.

"We're going to the edge of the forest---to the Acanthus River."

"Where---where are we?"
The hesitation said enough: she'd calculated the risk of revealing how vulnerable she was with that question 
and decided she was too desperate for the answer.

"What is your name?"

"Marion."
She held his gaze with a sort of unflinching steel that had him angling his head.

An answer for an answer.
He said, "We're in the middle of Adarlan.
You were about a day's hike from the Avery River."

Marion blinked.
He wondered if she even knew that---or had considered how she'd cross the mighty body of water that had claimed ships captained by the most seasoned of men and women.

She said, "Are we running, or can I sit for a moment?"

He listened to the sounds of the forest for any hint of danger, then jerked his chin.

Marion sighed as she sat on the moss and roots.
She surveyed him.
"I thought all the Fae were dead.
Even the demi-Fae."

"I'm from Wendlyn.
And you," he said, brows rising slightly, "are from Morath."

"Not from.
\emph{Escaping} from."

"Why---and how."

Her narrowed eyes told him enough: she knew he still didn't believe her, not entirely, red blood or no.
Yet she didn't answer, instead leaning over her legs to unlace a boot.
Her fingers trembled a bit, but she got through the laces, yanking off the boot, removing the sock, and rolling up her leather pant leg to reveal---

Shit.
He'd seen plenty of ruined bodies in his day, had done plenty of ruining himself, but rarely were they left so untreated.
Marion's leg was a mess of scar tissue and twisted bone.
And right above her misshapen ankle lay still-healing wounds where shackles had unmistakably been.

She said quietly, "Allies of Morath are usually whole.
Their dark magic could surely cure a cripple---and they surely would have no use for one."

That was why she'd managed so well with the limp.
She'd had years to master it, from the coloring of the scar tissue.

Marion rolled her pant leg back down but left her foot bare, massaging it.
She hissed through her teeth.

He sat on a fallen log a few feet away, taking off his own pack to rifle through it.
"Tell me what you know of Morath," he said, and chucked her a tin of salve straight from Doranelle.

The girl stared at it, those sharp eyes putting together what he was, where he was from, and what that tin likely contained.
When she lifted them to his face, she nodded silently in agreement of his offer: relief from the pain for answers.
She unscrewed the lid, and he caught the way her mouth parted as she breathed in the pungent herbs.

Pain and pleasure danced across her face as she began rubbing the salve into her old injuries.

And as she worked, she spoke.

Marion told him of the Ironteeth host, of the Wing Leader and the Thirteen, of the armies camped around the mountain Keep, of the places where only screaming echoed, of the countless forges and blacksmiths.
She described her own escape: without warning, she didn't know how, the castle had exploded.
She'd seen it as her chance, disguising herself in a witch's attire, grabbing one of their packs, and running.
In the chaos, no one had chased her.

"I've been running for weeks," she said.
"Apparently, I've barely covered half the distance."

"To where?"

Marion looked northward.
"Terrasen."

Lorcan stifled a snarl.
"You're not missing much."
"Have you news of it?"
Alarm filled those eyes.

"No," he said, shrugging.
She finished rubbing her foot and ankle.
"What's in Terrasen?
Your family?"
He had not asked why she'd been brought to Morath.
He didn't particularly care to hear her sad story.
Everyone had one, he'd found.

The girl's face tightened.
"I owe a debt to a friend---someone who helped me get out of Morath.
She bade me to find someone named Celaena Sardothien.
So that is my first task: learning who she is, where she is.
Terrasen seems like a better place to start than Adarlan."

No guile, no whisper of this meeting being anything but chance.

"And then," the girl went on, the brightness in her eyes growing, "I need to find Aelin Galathynius, the Queen of Terrasen."

It was an effort not to go for his sword.
"Why?"

Marion glanced toward him, as if she'd somehow even forgotten he was there.
"I heard a rumor that she's raising an army to stop the one in Morath.
I plan to offer my services."

"Why?"
he said again.
Aside from the wits that had kept her out of the ilken's claws, he saw no other reason for the bitch-queen to need the girl.

Marion's full mouth tightened.
"Because I am from Terrasen and believed my queen dead.
And now she is alive, and fighting, so I will fight with her.
So that no other girls will be taken from their homes and brought to Morath and forgotten."

Lorcan debated telling her what he knew: that her two quests were one and the same.
But that would lead to questions from her, and he was in no mood---

"Why do you wish to go to Morath?
Everyone else is fleeing from it."

"I was sent by my mistress to stop the threat it poses."

"You're one man---male."
Not an insult, but Lorcan stared her down anyway.

"I have my skills, just as you have yours."

Her eyes darted to his hands, now crusted in dried black blood.
He wondered, though, if she was imagining the magic that had sparked there.

He waited for Marion to ask more, but she pulled on her sock, then her boot, and laced it up.
"We shouldn't rest for long."
Indeed.

She eased to her feet, wincing a bit, but gave an appreciative frown toward her leg.
Lorcan took that as answer enough regarding the salve's efficiency.
She bent down to retrieve the tin, her dark curtain of hair sweeping over her face.
At some point, it had come free of its braid.

She rose, chucking him the tin.
He caught it in one hand.
"Once we reach the Acanthus, what then?"

He pocketed the tin in his cloak.
"There are countless merchants' caravans and seasonal carnivals wandering the plains---I passed many on my way down here.
Some might even be trying to cross the river.
We'll get in with one of them.
Hide out.
Once we've crossed and wandered far enough onto the grasslands, you'll take one north; I'll head south."

Her eyes narrowed slightly.
But Marion said, "Why travel with me at all?"

"There are more details regarding Morath's interior that I want from you.
I'll keep you from danger, and you'll provide them for me."

The sun began its final descent, bathing the woods in gold.
Marion frowned slightly.
"You swear it?
That you will protect me?"

"I didn't leave you to the ilken today, did I?"

She eyed him with a clarity and frankness that made him pause.
"Swear it."

He rolled his eyes.
"I promise."
The girl had no idea that for the past five centuries, promises were the only currency he really traded in.
"I will not abandon you."

She nodded, seemingly satisfied with that.
"Then I will tell you what I know."

He started eastward, slinging his pack over his shoulder.

But Marion said, "They'll be hunting for us at every crossing, searching wagons.
If they could find me here, they'll find me on any main road."
And find him, too, if the witches were still out for his blood.

Lorcan said, "And you have some idea around this?"

A faint smile danced around her rosebud mouth, despite the horrors they'd escaped, her misery in the woods.
"I might."
