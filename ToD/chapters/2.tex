
\chapter{2}

There were two parts of her, Nesryn supposed.

The part that was now Captain of Adarlan's Royal Guard, who had made a vow to her king to see that the man in the wheeled chair beside her was healed--- and to muster an army from the man enthroned before her.
That part of Nesryn kept her head high, her shoulders back, her hands within a nonthreatening distance of the ornate sword at her hip.

Then there was the other part.

The part that had glimpsed the spires and minarets and domes of the god-city breaking over the horizon as they'd sailed in, the shining pillar of the Torre standing proud over it all, and had to swallow back tears.
The part that had scented the smoky paprika and crisp tang of ginger and beckoning sweetness of cumin as soon as she had cleared the docks and knew, deep in her bones, that she was \emph{home}.
That, yes, she lived and served and would die for Adarlan, for the family still there, but this place, where her father had once lived and where even her Adarlan-born mother had felt more at ease\ldots These were her people.

The skin in varying shades of brown and tan.
The abundance of that shining black hair---\emph{her} hair.
The eyes that ranged from uptilted to wide and round to slender, in hues of ebony and chestnut and even the rare hazel and green.
Her people.
A blend of kingdoms and territories, yes, but\ldots Here there were no slurs hissed in the streets.
Here there would be no rocks thrown by children.
Here her sister's children would not feel different.
Unwanted.

And that part of her\ldots Despite her thrown-back shoulders and raised chin, her knees indeed quaked at who---at \emph{what}---stood before her.

Nesryn had not dared tell her father where and what she was leaving to do.
Only that she was off on an errand of the King of Adarlan and would not be back for some time.

Her father wouldn't have believed it.
Nesryn didn't quite believe it herself.

The khagan had been a story whispered before their hearth on winter nights, his offspring legends told while kneading endless loaves of bread for their bakery.
Their ancestors' bedside tales to either lull her into sweet sleep or keep her up all night in bone-deep terror.

The khagan was a living myth.
As much of a deity as the thirty-six gods who ruled over this city and empire.

There were as many temples to those gods in Antica as there were tributes to the various khagans.
\emph{More}.

They called it the god-city for them---and for the living god seated on the ivory throne atop that golden dais.

It was indeed pure gold, just as her father's whispered legends claimed.

And the khagan's six children\ldots Nesryn could name them all without introduction.

After the meticulous research Chaol had done while on their ship, she had no doubt he could as well.

But that was not how this meeting was to go.

For as much as \emph{she} had taught the former captain about her homeland these weeks, he'd instructed her on court protocol.
He had rarely been so directly involved, yes, but he had witnessed enough of it while serving the king.

An observer of the game who was now to be a prime player.
With the stakes unbearably high.

They waited in silence for the khagan to speak.

She'd tried not to gawk while walking through the palace.
She had never set foot inside it during her few visits to Antica over the years.
Neither had her father, or his father, or any of her ancestors.
In a city of gods, this was the holiest of temples.
And deadliest of labyrinths.

The khagan did not move from his ivory throne.

A newer, wider throne, dating from a hundred years ago---when the seventh khagan had chucked out the old one because his large frame didn't fit in it.
He'd eaten and drunk himself to death, history claimed, but at least had the good sense to name his Heir before he clutched his chest one day and slumped dead\ldots right in that throne.

Urus, the current khagan, was no more than sixty, and seemed in far better condition.
Though his dark hair had long since gone as white as his carved throne, though scars peppered his wrinkled skin as a reminder to all that \emph{he} had fought for this throne in the final days of his mother's life\ldots His onyx eyes, slender and uptilted, were bright as stars.
Aware and all-seeing.

Atop his snowy head sat no crown.
For gods among mortals did not need markers of their divine rule.

Behind him, strips of white silk tied to the open windows fluttered in the hot breeze.
Sending the thoughts of the khagan and his family to where the soul of the deceased---whoever they might be, someone important, no doubt---had now rejoined the Eternal Blue Sky and Slumbering Earth that the khagan and all his ancestors still honored in lieu of the pantheon of thirty-six gods their citizens remained free to worship.

Or any other gods outside of it, should their territories be new enough to not yet have had their gods incorporated into the fold.
There had to be several of those, since during his three decades of rule, the man seated before them had added a handful of overseas kingdoms to their borders.

A kingdom for every ring adorning his scar-flecked fingers, precious stones glinting among them.

A warrior bedecked in finery.
Those hands slid from the arms of his ivory throne---assembled from the hewn tusks of the mighty beasts that roamed the central grasslands---and settled in his lap, hidden beneath swaths of goldtrimmed blue silk.
Indigo dye from the steamy, lush lands in the west.
From Balruhn, where Nesryn's own people had originally hailed, before curiosity and ambition drove her great-grandfather to drag his family over mountains and grasslands and deserts to the god-city in the arid north.

The Faliqs had long been tradesmen, and not of anything particularly fine.
Just simple, good cloth and household spices.
Her uncle still traded such things and, through various lucrative investments, had become a moderately wealthy man, his family now dwelling in a beautiful home within this very city.
A definitive step up from a baker---the path her father had chosen upon leaving these shores.

"It is not every day that a new king sends someone so important to our shores," the khagan said at last, using their own tongue and not Halha, the language of the southern continent.
"I suppose we should deem it an honor."

His accent was so like her father's---but the tone lacked the warmth, the humor.
A man who had been obeyed his entire life, and fought to earn his crown.
And executed two of the siblings who proved to be sore losers.
The surviving three\ldots one had gone into exile, and the other two had sworn fealty to their brother.
By having the healers of the Torre render them infertile.

Chaol inclined his head.
"The honor is mine, Great Khagan."

Not \emph{Majesty}---that was for kings or queens.
There was no term high or grand enough for this man before them.
Only the title that the first of his ancestors had borne: Great Khagan.

"Yours," the khagan mused, those dark eyes now sliding to Nesryn.
"And what of your companion?"

Nesryn fought the urge to bow again.
Dorian Havilliard was the opposite of this man, she realized.
Aelin Galathynius, however\ldots Nesryn wondered if the young queen might have more in common with the khagan than she did with the Havilliard king.
Or would, if Aelin survived long enough.
If she reached her throne.

Nesryn shoved those thoughts down as Chaol peered at her, his shoulders tightening.
Not at the words, not at the company, but simply because she knew that the mere act of having to look \emph{up}, facing this mighty warrior-king in that chair\ldots Today would be a hard one for him.

Nesryn inclined her head slightly.
"I am Nesryn Faliq, Captain of the Royal Guard of Adarlan.
As Lord Westfall once was before King Dorian appointed him as his Hand earlier this summer."
She was grateful that years spent living in Rifthold had taught her not to smile, not to cringe or show fear---grateful that she'd learned to keep her voice cool and steady even while her knees quaked.

Nesryn continued, "My family hails from here, Great Khagan.
Antica still owns a piece of my soul."
She placed a hand over her heart, the fine threads of her gold-and-crimson uniform, the colors of the empire that had made her family often feel hunted and unwanted, scraping against her calluses.
"The honor of being in your palace is the greatest of my life."
It was, perhaps, true.

If she found time to visit her family in the quiet, garden-filled Runni Quarter ---home mostly to merchants and tradesmen like her uncle---they would certainly consider it so.

The khagan only smiled a bit.
"Then allow me to welcome you to your true home, Captain."

Nesryn felt, more than saw, Chaol's flicker of annoyance.
She wasn't entirely certain what had triggered it: the claim on her homeland, or the official title that had now passed to her.

But Nesryn bowed her head again in thanks.

The khagan said to Chaol, "I will assume you are here to woo me into joining this war of yours."

Chaol countered a shade tersely, "We're here at the behest of my king."
A note of pride at that word.
"To begin what we hope will be a new era of prosperous trade and peace."

One of the khagan's offspring---a young woman with hair like flowing night and eyes like dark fire---exchanged a wry look with the sibling to her left, a man perhaps three years her elder.

Hasar and Sartaq, then.
Third and secondborn, respectively.
Each wore similar loose pants and embroidered tunics, with fine leather boots rising to their knees.
Hasar was no beauty, but those eyes\ldots The flame dancing in them as she glanced to her elder brother made up for it.

And Sartaq---commander of his father's ruk riders.
The rukhin.

The northern aerial cavalry of his people had long dwelled in the towering Tavan Mountains with their ruks: enormous birds, eagle-like in shape, large enough to carry off cattle and horses.
Without the sheer bulk and destructive weight of the Ironteeth witches' wyverns, but swift and nimble and clever as foxes.
The perfect mounts for the legendary archers who flew them into battle.

Sartaq's face was solemn, his broad shoulders thrown back.
A man perhaps as ill at ease in his fine clothes as Chaol.
She wondered if his ruk, Kadara, was perched on one of the palace's thirty-six minarets, eyeing the cowering servants and guards, waiting impatiently for her master's return.

That Sartaq was here\ldots They had to have known, then.
Well in advance.
That she and Chaol were coming.

The knowing glance that passed between Sartaq and Hasar told Nesryn enough: they, at least, had discussed the possibilities of this visit.

Sartaq's gaze slid from his sister to Nesryn.

She yielded a blink.
His brown skin was darker than the others'---perhaps from all that time in the skies and sunlight---his eyes a solid ebony.
Depthless and unreadable.
His black hair remained unbound save for a small braid that curved over the arch of his ear.
The rest of his hair fell to just past his muscled chest, and swayed slightly as he gave what Nesryn could have sworn was a mocking incline of his head.

A ragtag, humbled pair, Adarlan had sent.
The injured former captain, and the common-bred current one.
Perhaps the khagan's initial words about \emph{honor} had been a veiled mention of what he perceived as an insult.

Nesryn dragged her attention away from the prince, even as she felt Sartaq's keen stare lingering like some phantom touch.

"We arrive bearing gifts from His Majesty, the King of Adarlan," Chaol was saying, and twisted in his chair to motion the servants behind them to come forward.

Queen Georgina and her court had practically raided the royal coffers before they'd fled to their mountain estates this spring.
And the former king had smuggled out much of what was left during those final few months.
But before they'd sailed here, Dorian had ventured into the many vaults beneath the castle.
Nesryn still could hear his echoed curse, filthier than she'd ever heard him speak, as he found little more than gold marks within.

Aelin, as usual, had a plan.

Nesryn had been standing beside her new king when Aelin had flipped open two trunks in her chambers.
Jewelry fit for a queen---for a Queen of Assassins--- had sparkled within.

\emph{I've enough funds for now}, Aelin had only said to Dorian when he began to object.
\emph{Give the khagan some of Adarlan's finest.}

In the weeks since, Nesryn had wondered if Aelin had been glad to be rid of what she'd purchased with her blood money.
The jewels of Adarlan, it seemed, would not travel to Terrasen.

And now, as the servants laid out the four smaller trunks---divided from the original two to make it seem like \emph{more}, Aelin had suggested---as they flipped open the lids, the still-silent court pressed in to see.

A murmur went through them at the glistening gems and gold and silver.

"A gift," Chaol declared as even the khagan himself leaned forward to examine the trove.
"From King Dorian Havilliard of Adarlan, and Aelin Galathynius, Queen of Terrasen."

Princess Hasar's eyes snapped to Chaol at the second name.

Prince Sartaq only glanced back at his father.
The eldest son, Arghun, frowned at the jewels.

Arghun---the politician amongst them, beloved by the merchants and power brokers of the continent.
Slender and tall, he was a scholar who traded not in coin and finery but in knowledge.

\emph{Prince of Spies}, they called Arghun.
While his two brothers had become the finest of warriors, Arghun had honed his mind, and now oversaw his father's thirty-six viziers.
So that frown at the treasure Necklaces of diamond and ruby.
Bracelets of gold and emerald.
Earrings--- veritable small chandeliers---of sapphire and amethyst.
Exquisitely wrought rings, some crowned with jewels as large as a swallow's egg.
Combs and pins and brooches.
Blood-gained, blood-bought.

The youngest of the assembled royal children, a fine-boned, comely woman, leaned the closest.
Duva.
A thick silver ring with a sapphire of near-obscene size adorned her slender hand, pressed delicately against the considerable swell of her belly.

Perhaps six months along, though the flowing clothes---she favored purple and rose---and her slight build could distort that.
Certainly her first child, the result of her arranged marriage to a prince hailing from an overseas territory to the far east, a southern neighbor of Doranelle that had noted the rumblings of its Fae Queen and wanted to secure the protection of the southern empire across the ocean.
Perhaps the first attempt, Nesryn and others had wondered, of the khaganate greatly expanding its own considerable continent.

Nesryn didn't let herself look too long at the life growing beneath that bejeweled hand.

For if one of Duva's siblings were crowned khagan, the first task of the new ruler---after his or her sufficient offspring were produced---would be to eliminate any other challenges to the throne.
Starting with the offspring of his or her siblings, if they challenged their right to rule.

She wondered how Duva was able to endure it.
If she had come to love the babe growing in her womb, or if she was wise enough to not allow such a feeling.
If the father of that babe would do everything he could to get that child to safety should it come to that.

The khagan at last leaned back in his throne.
His children had straightened again, Duva's hand falling back at her side.

"Jewels," Chaol explained, "set by the finest of Adarlanian craftsmen."

The khagan toyed with a citrine ring on his own hand.
"If they came from Aelin Galathynius's trove, I have no doubt that they are."

A beat of silence between Nesryn and Chaol.
They had known---anticipated ---that the khagan had spies in every land, on every sea.
That Aelin's past might be just a tad difficult to work around.

"For you are not only Adarlan's Hand," the khagan went on, "but also the Ambassador of Terrasen, are you not?"
"Indeed I am," Chaol said simply.

The khagan rose with only the slightest stiffness, his children immediately stepping aside to clear a path for him to step off the golden dais.

The tallest of them---strapping and perhaps more unchecked than Sartaq's quiet intensity---eyed up the crowd as if assessing any threats within.
Kashin.
Fourthborn.

If Sartaq commanded the ruks in the northern and central skies, then Kashin controlled the armies on land.
Foot soldiers and the horse-lords, mostly.
Arghun held sway over the viziers, and Hasar, rumor claimed, had the armadas bowing to her.
Yet there remained something less polished about Kashin, his dark hair braided back from his broad-planed face.
Handsome, yes---but it was as if life amongst his troops had rubbed off on him, and not necessarily in a bad way.

The khagan descended the dais, his cobalt robes whispering along the floor.
And with every step over the green marble, Nesryn realized that this man had indeed once commanded not just the ruks in the skies, but also the horse-lords, \emph{and} swayed the armadas to join him.
And then Urus and his elder brother had gone hand-to-hand in combat at the behest of their mother while she lay dying from a wasting sickness that even the Torre could not heal.
The son who walked off the sand would be khagan.

The former khagan had a penchant for spectacle.
And for this final fight between her two selected offspring, she had placed them in the great amphitheater in the heart of the city, the doors open to any who could claw inside to find a seat.
People had sat upon the archways and steps, with thousands cramming the streets that flowed to the white-stoned building.
Ruks and their riders had perched on the pillars crowning the uppermost level, more rukhin circling in the skies above.

The two would-be Heirs had fought for six hours.

Not just against each other, but also against the horrors their mother unleashed to test them: great cats sprang from hidden cages beneath the sandy floor; iron-spiked chariots with spear-throwers had charged from the gloom of the tunnel entrances to run them down.

Nesryn's father had been amongst the frenzied mob in the streets, listening to the shouted reports from those dangling off the columns.
The final blow hadn't been an act of brutality or hate.

The now-khagan's elder brother, Orda, had taken a spear to the side thanks to one of those charioteers.
After six hours of bloody battle and survival, the blow had kept him down.

And Urus had set aside his sword.
Absolute silence had fallen in the arena.

Silence as Urus had extended a bloodied hand to his fallen brother---to help him.

Orda had sent a hidden dagger shooting for Urus's heart.

It had missed by two inches.

And Urus had ripped that dagger free, screaming, and plunged it right back into his brother.

Urus did not miss as his brother had.

Nesryn wondered if a scar still marred the khagan's chest as he now strode toward her and Chaol and the jewels displayed.
If that long-dead khagan had wept for her fallen son in private, slain by the one who would take her crown in a matter of days.
Or if she had never allowed herself to love her children, knowing what must befall them.

Urus, Khagan of the Southern Continent, stopped before Nesryn and Chaol.
He towered over Nesryn by a good half foot, his shoulders still broad, spine still straight.

He bent with only a touch of age-granted strain to pluck up a necklace of diamond and sapphire from the chest.
It glittered like a living river in his scarflecked, bejeweled hands.

"My eldest, Arghun," said the khagan, jerking his chin toward the narrowfaced prince monitoring all, "recently informed me of some fascinating information regarding Queen Aelin Ashryver Galathynius."

Nesryn waited for the blow.
Chaol just held Urus's gaze.

But the khagan's dark eyes---Sartaq's eyes, she realized---danced as he said to Chaol, "A queen at nineteen would make many uneasy.
Dorian Havilliard, at least, has been trained since birth to take up his crown, to control a court and kingdom.
But Aelin Galathynius \ldots"

The khagan chucked the necklace into the chest.
Its thunk was as loud as steel on stone.

"I suppose some would call ten years as a trained assassin to be experience."

Murmurs again rippled through the throne room.
Hasar's fire-bright eyes practically glowed.
Sartaq's face did not shift at all.
Perhaps a skill learned from his eldest brother---whose spies had to be skilled indeed if they'd learned of Aelin's past.
Even though Arghun himself seemed to be struggling to keep a smug smile from his lips.

"We may be separated by the Narrow Sea," the khagan said to Chaol, whose features did not so much as alter, "but even we have heard of Celaena Sardothien.
You bring me jewels, no doubt from her own collection.
Yet they are jewels for \emph{me}, when my daughter Duva"---a glance toward his pregnant, pretty daughter standing closely beside Hasar---"has yet to receive any sort of wedding gift from either your new king or returned queen, while every other ruler sent theirs nearly half a year ago."

Nesryn hid her wince.
An oversight that could be explained by so many truths---but not ones that they dared voice, not here.
Chaol didn't offer any of them as he remained silent.

"But," the khagan went on, "regardless of the jewels you've now dumped at my feet like sacks of grain, I would still rather have the truth.
Especially after Aelin Galathynius shattered your own glass castle, murdered your former king, and seized your capital city."

"If Prince Arghun has the information," Chaol said at last with unfaltering coolness, "perhaps you do not need it from me."

Nesryn stifled her cringe at the defiance, the tone---

"Perhaps not," the khagan said, even as Arghun's eyes narrowed slightly.

"But I think \emph{you} should like some truth from me."

Chaol didn't ask for it.
Didn't look remotely interested beyond his,"Oh?"

Kashin stiffened.
His father's fiercest defender, then.
Arghun only exchanged glances with a vizier and smiled toward Chaol like an adder ready to strike.

"Here is why I think you have come, Lord Westfall, Hand to the King."

Only the gulls wheeling high above the dome of the throne room dared make any noise.

The khagan shut lid after lid on the trunks.

"I think you have come to convince me to join your war.
Adarlan is cleaved, Terrasen is destitute, and will no doubt have some issue convincing her surviving lords to fight for an untried queen who spent ten years indulging herself in Rifthold, purchasing these jewels with blood money.
Your list of allies is short and brittle.
Duke Perrington's forces are anything but.
The other kingdoms on your continent are shattered and separated from your northern territories by Perrington's armies.
So you have arrived here, fast as the eight winds can carry you, to beg me to send my armies to your shores.
To convince me to spill our blood on a lost cause."

"Some might consider it a noble cause," Chaol countered.

"I am not done yet," the khagan said, lifting a hand.

Chaol bristled but did not speak out of turn again.
Nesryn's heart thundered.

"Many would argue," the khagan said, waving that upraised hand toward a few viziers, toward Arghun and Hasar, "that we remain out of it.
Or better yet, ally with the force sure to win, whose trade has been profitable for us these ten years."

A wave of that hand toward some other men and women in the gold robes of viziers.
Toward Sartaq and Kashin and Duva.
"Some would say that we risk allying with Perrington only to potentially face his armies in our harbors one day.
That the shattered kingdoms of Eyllwe and Fenharrow might again become wealthy under new rule, and fill our coffers with good trade.
I have no doubt you will promise me that it shall be so.
You will offer me exclusive trading deals, likely to your own disadvantage.
But you are desperate, and there is nothing you possess that I do not already own.
That I cannot take if I wish."

Chaol kept his mouth shut, thankfully.
Even as his brown eyes simmered at the quiet threat.

The khagan peered into the fourth and final trunk.
Jeweled combs and brushes, ornate perfume bottles made by Adarlan's finest glassblowers.
The same who had built the castle Aelin had shattered.
"So, you have come to convince me to join your cause.
And I shall consider it while you stay here.

Since you have undoubtedly come for another purpose, too."

A flick of that scarred, jeweled hand toward the chair.
Color stained Chaol's tan cheeks, but he did not flinch, did not cower.
Nesryn forced herself to do the same.

"Arghun informed me your injuries are new---that they happened when the glass castle exploded.
It seems the Queen of Terrasen was not quite so careful about shielding her allies."

A muscle feathered in Chaol's jaw as everyone, from prince to servant, looked to his legs.

"Because your relations with Doranelle are now strained, also thanks to Aelin Galathynius, I assume the only path toward healing that remains open to you is here.
At the Torre Cesme."

The khagan shrugged, the only reveal of the irreverent warrior-youth he'd once been.
"My beloved wife will be deeply upset if I were to deny an injured man a chance at healing"---the empress was nowhere to be seen in this room, Nesryn realized with a start---"so I, of course, shall grant you permission to enter the Torre.
Whether its healers will agree to work upon you shall be up to them.
Even I do not control the will of the Torre."

The Torre---the Tower.
It dominated the southern edge of Antica, nestled atop its highest hill to overlook the city that sloped down toward the green sea.
Domain of its famed healers, and tribute to Silba, the healer-goddess who blessed them.
Of the thirty-six gods this empire had welcomed into the fold over the centuries, from religions near and far, in this city of gods\ldots Silba reigned unchallenged.

Chaol looked like he was swallowing hot coals, but he mercifully managed to bow his head.
"I thank you for your generosity, Great Khagan."

"Rest tonight---I will inform them that you shall be ready tomorrow morning.
Since you cannot go to them, one will be sent to you.
If they agree."

Chaol's fingers shifted in his lap, but he did not clench them.
Nesryn still held her breath.

"I am at their disposal," Chaol said tightly.

The khagan shut the final trunk of jewels.
"You may keep your presents, Hand of the King, Ambassador to Aelin Galathynius.
I have no use for them--- and no interest."

Chaol's head snapped up, as if something in the khagan's tone had snared him.
"Why."

Nesryn barely hid her cringe.
More of a demand than anyone ever dared make of the man, judging by the surprised anger in the khagan's eyes, in the glances exchanged between his children.

But Nesryn caught the flicker of something else within the khagan's eyes.
A weariness.

Something oily slid into her gut as she noted the white banners streaming from the windows, all over the city.
As she looked to the six heirs and counted again.

Not six.

Five.
Only five were here.

Death-banners at the royal household.
All over the city.

They were not a mourning people---not in the way they could be in Adarlan, dressing all in black and moping for months.
Even amongst the khagan's royal family, life picked up and went on, their dead not stuffed in stone catacombs or coffins, but shrouded in white and laid beneath the open skies of their sealed-off, sacred reserve on the distant steppes.

Nesryn glanced down the line of five heirs, counting.
The eldest five were present.
And just as she realized that Tumelun, the youngest---barely seventeen ---was not there, the khagan said to Chaol, "Your spies are indeed useless if you have not heard."

With that, he strode for his throne, leaving Sartaq to step forward, the secondeldest prince's depthless eyes veiled with sorrow.
Sartaq gave Nesryn a silent nod.
Yes.
Yes, her suspicions were right---

Sartaq's solid, pleasant voice filled the chamber.
"Our beloved sister,

Tumelun, died unexpectedly three weeks ago."

Oh, gods.
So many words and rituals had been passed over; merely coming here to demand their aid in war was uncouth, untoward---

Chaol said into the fraught silence, meeting the stares of each taut-faced prince and princess, then finally the weary-eyed khagan himself, "You have my deepest condolences."

Nesryn breathed, "May the northern wind carry her to fairer plains."

Only Sartaq bothered to nod his thanks, while the others now turned cold and stiff.

Nesryn shot Chaol a silent, warning look not to ask about the death.
He read the expression on her face and nodded.

The khagan scratched at a fleck on his ivory throne, the silence as heavy as one of the coats the horse-lords still wore against that bitter northern wind on the steppes and their unforgiving wooden saddles.

"We've been at sea for three weeks," Chaol tried to offer, his voice softer now.

The khagan did not bother to appear understanding.
"That would also explain why you are so unaware of the other bit of news, and why these cold jewels might be of more use for \emph{you}."
The khagan's lips curled in a mirthless smile.
"Arghun's contacts also brought word from a ship this morning.
Your royal coffers in Rifthold are no longer accessible.
Duke Perrington and his host of flying terrors have sacked Rifthold."

Silence, pulsing and hollow, swept through Nesryn.
She wasn't sure if Chaol was breathing.

"We do not have word on King Dorian's location, but he yielded Rifthold to them.
Fled into the night, if rumor is to be believed.
The city has fallen.

Everything to the south of Rifthold belongs to Perrington and his witches now."
Nesryn saw the faces of her nieces and nephews first.

Then the face of her sister.
Then her father.
Saw their kitchen, the bakery.
The pear tarts cooling on the long, wooden table.

Dorian had left them.
Left them all to\ldots to do what?
Find help?
Survive?
Run to Aelin?

Had the royal guard remained to fight?
Had anyone fought to save the innocents in the city?

Her hands were shaking.
She didn't care.
Didn't care if these people clad in riches sneered.

Her sister's children, the great joy in her life\ldots Chaol was staring up at her.
Nothing on his face.
No devastation, no shock.

That crimson-and-gold uniform became stifling.
Strangling.

Witches and wyverns.
In her city.
With those iron teeth and nails.
Shredding and bleeding and tormenting.
Her family---her \emph{family}---

"Father."

Sartaq had stepped forward once more.
Those onyx eyes slid between Nesryn and the khagan.
"It has been a long journey for our guests.
Politics aside," he said, giving a disapproving glance at Arghun, who seemed amused---\emph{amused} at this news he'd brought, that had set the green marble floors roiling beneath her boots---"we are still a nation of hospitality.
Let them rest for a few hours.
And then join us for dinner."

Hasar came to Sartaq's side, frowning at Arghun while she did.
Perhaps not from reprimand like her brother, but simply for Arghun not telling \emph{her} of this news first.
"Let no guest pass through our home and find its comforts lacking."
Even though the words were welcoming, Hasar's tone was anything but.

Their father gave them a bemused glance.
"Indeed."
Urus waved a hand toward the servants by the far pillars.
"Escort them to their rooms.
And dispatch a message to the Torre to send their finest---Hafiza, if she'll come down from that tower."

Nesryn scarcely heard the rest.
If the witches held the city, then the Valg who had infested it earlier this summer\ldots There would be no one to fight them.
No one to shield her family.

If they had survived.

She couldn't breathe.
Couldn't think.

She should not have left.
Should not have taken this position.

They could be dead, or suffering.
Dead.
Dead.

She did not notice the female servant who came to push Chaol's chair.
Barely noticed the hand Chaol reached out to twine through her own.

Nesryn didn't so much as bow to the khagan as they left.

She could not stop seeing their faces.

The children.
Her sister's smiling, round-bellied children.

She should not have come.

