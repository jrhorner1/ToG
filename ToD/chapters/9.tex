
\chapter{9}

Two hours later, her head leaning against the lip of the tub carved into the stone floor of the enormous cavern beneath the Torre, Yrene stared into the darkness lurking high above.

The Womb was nearly empty in the midafternoon. Her only company was the trickle of the natural hot spring waters flowing through the dozen tubs built into the cave floor, and the drip of water from jagged stalactites landing upon the countless bells strung on chains between the pillars of pale stone that rose up from the ancient rock.

Candles had been tucked into natural alcoves, or had been clumped at either end of each sunken tub, gilding the sulfurous steam and setting the owls carved into every wall and slick pillar in flickering relief.

A plush cloth cushioning her head against the unforgiving stone lip of the tub, Yrene breathed in the Womb's thick air, watching it rise and vanish into the clear, crisp darkness squatting far overhead. All around her echoed high-pitched, sweet ringing, occasionally interrupted by solitary clear notes.

No one in the Torre knew who had first brought the various bells of silver and glass and bronze down to the open chamber of Silba's Womb. Some bells had been there so long they were crusted with mineral deposits, their ringing as water dropped from the stalactites now no more than a faint \emph{plunk}. But it was tradition ---one Yrene herself had participated in---for each new acolyte to bring a bell of her choosing. To have her name and date of entry into the Torre engraved on it, and to then find a place for it, before she first immersed herself in the bubbling waters of the Womb floor. The bell to hang for eternity, offering music and guidance to all healers who came afterward;
the voices of their beloved sisters forever singing to them.

And considering how many healers had passed through the Torre halls, considering the number of bells, large and small, that now hung throughout the space  The entire chamber, nearly the size of the khagan's great hall, was full of the echoing, layered ringing. A steady hum that filled Yrene's head, her bones, as she soaked in the delicious heat.

Some ancient architect had discovered the hot springs far beneath the Torre and constructed a network of tubs built into the floor so that the water flowed between them, a constant stream of warmth and movement. Yrene held her hand against one of the vents in the side of the tub, letting the water ripple through her fingers on its way to the vent on the other end, to pass back into the stream itself ---and into the slumbering heart of the earth.

Yrene took another deep breath, brushing back the damp hair clinging to her brow. She'd washed before entering the tub, as all were required to do in one of the small antechambers outside the Womb, to clean away the dust and blood and stains of the world above. An acolyte had been waiting with a lightweight robe of lavender---Silba's color---for Yrene to wear into the Womb proper, where she'd discarded it beside the pool and stepped in, naked save for her mother's ring.

In the curling steam, Yrene lifted her hand before her and studied the ring, the way the light bent along the gold and smoldered in the garnet. All around, bells rang and hummed and sang, blending with the trickling water until she was adrift in a stream of living sound.

Water---Silba's element. To bathe in the sacred waters here, untouched by the world above, was to enter Silba's very lifeblood. Yrene knew she was not the only healer who had taken the waters and felt as if she were indeed nestled in the warmth of Silba's womb. As if this space had been made for them alone.

And the darkness above her  it was different from what she had spied in Lord Westfall's body. The opposite of that blackness. The darkness above her was that of creation, of rest, of unformed thought.

Yrene stared into it, into the womb of Silba herself. And could have sworn she felt something staring back. Listening, while she thought through all Lord Westfall had told her.

Things out of ancient nightmares. Things from another realm. Demons. Dark magics. Poised to unleash themselves upon her homeland. Even in the soothing, warm waters, Yrene's blood chilled.

On those northern, far-off battlefields, she had expected to treat stab wounds and arrows and shattered bones. Expected to treat any of the diseases that ran rampant in army camps, especially during the colder months.

Not wounds from creatures that destroyed soul as well as body. That used talons and teeth and poison. The maleficent power coiled around the injury to his spine  It was not some fractured bone or tangled-up nerves. Well, it technically \emph{was}, but that fell magic was tied to it. Bound to it.

She still could not shake the oily feel, the sense that something inside it had stirred. Awoken.

The ringing of the bells flowed and ebbed, lulling her mind to rest, to open.

She'd go to the library tonight. See if there was any information regarding all the lord had claimed, if perhaps someone before her had any thoughts on magically granted injuries.

Yet it would not be an injury that solely relied upon her to heal.

She'd suggested as much before leaving. But to battle that thing within him

 \emph{How?}

Yrene mouthed the word into the steam and dark, into the ringing, bubbling quiet.

She could still see her probe of magic recoiling, still feel its repulsion from that demon-born power. The opposite of what she was, what her magic was. In the darkness hovering overhead, she could see it all. In the darkness far above, tucked into Silba's earthly womb  it beckoned.

As if to say, \emph{You must enter where you fear to tread.}

Yrene swallowed. To delve into that festering pit of power that had latched itself onto the lord's back 

\emph{You must enter}, the sweet darkness whispered, the water singing along with it while it flowed around and past her. As if she were swimming in Silba's veins.

\emph{You must enter}, it murmured again, the darkness above seeming to spread, to inch closer.

Yrene let it. And let herself stare deeper, move deeper, into that dark.

To fight that festering force within the lord, to risk it for some test of Hafiza's, to risk it for a son of Adarlan when her own people were being attacked or battling in that distant war and every day delayed her
 \emph{I can't.}

\emph{You won't}, the lovely darkness challenged.

Yrene balked. She had promised Hafiza to remain, to heal him, but what she'd felt today  It could take an untold amount of time. If she could even find a way to help him. She'd promised to heal him, and though some injuries required the healer to walk the road with their patient, \emph{this} injury of his--- The darkness seemed to recede.

\emph{I can't}, Yrene insisted.

It did not answer again. Distantly, as if she were now far away, a bell rang, clear and pure.

Yrene blinked at the sound, the world tumbling into focus. Her limbs and breath returning, as if she'd drifted above them.

She peered at the darkness---finding only smooth, veiling black. Hollow and empty, as if it had been vacated. There, and gone. As if she had repelled it, disappointed it.

Yrene's head spun slightly as she sat up, stretching limbs that had gone a bit stiff, even in the mineral-rich water. How long had she soaked?

She rubbed at her slick arms, heart thundering as she scanned the darkness, as if it might still have another answer for what she must do, what lay before her.

An alternative.

None came.

A sound shuffled through the cavern, distinctly not ringing or trickling or lapping. A quiet, shuddering intake of breath.

Yrene turned, water dripping off the errant strands of hair that had escaped the knot atop her head, and found another healer had entered the Womb at some point, claiming a tub on the opposite end of the parallel rows flanking either side of the chamber. With the drifting veils of steam, it was nearly impossible to identify her, though Yrene certainly didn't know the name of every healer in the Torre.

The sound rasped through the Womb again, and Yrene sat up farther, hands bracing on the cool, dark floor as she stood from the water. Steam curled off her skin as she reached for the thin robe and tied it around her, the fabric clinging to her soaked body.

The Womb's protocol was well established. It was a place for solitude, for silence. Healers entered the waters to reconnect with Silba, to center themselves. Some sought guidance; some sought absolution; some sought to release a hard day's worth of emotions they could not show before patients, perhaps could not show before anyone.

And though Yrene knew the healer across the Womb was entitled to her space, though she was prepared to leave and grant the healer privacy to weep  The woman's shoulders shook. Another muffled sob.

On near-silent feet, Yrene approached the healer in the tub. Saw the rivulets down her young face---her light brown skin and gold-kissed umber hair nearly identical to Yrene's own. Saw the bleakness in the woman's tawny eyes as she gazed at the darkness high above, tears dripping off her slender jaw and into the rippling water.

There were some wounds that could not be healed. Some illnesses that even the healers' power could not stop, if rooted too deeply. If they had come too late. If they did not mark the right signs.

The healer did not look at her as Yrene silently sat beside her tub, curling her knees to her chest before she picked up the healer's hand and interlaced their fingers.

So Yrene sat there, holding the healer's hand while she silently wept, the drifting steam full of the clear, sweet ringing of those bells.

After untold minutes, the woman in the tub murmured, "She was three years old."

Yrene squeezed the healer's damp hand. There were no words to comfort, to soothe.

"I wish \ldots" The woman's voice broke, her entire body shaking, candlelight jumping along her beige skin. "Sometimes I wish this gift had never been given to me."

Yrene stilled at the words.

The woman at last turned her head, scanning Yrene's face, a flicker of recognition in her eyes. "Do you ever feel that way?" A raw, unguarded question.

No. She hadn't. Not once. Not even when the smoke of her mother's immolation had stung her eyes and she knew she could do nothing to save her. She had never once hated the gift she'd been given, because in all those years, she had never been alone thanks to it. Even with magic gone in her homeland, Yrene had still felt it, like a warm hand clasping her shoulder. A reminder of who she was, where she had come from, a living tether to countless generations of Towers women who had walked this path before her.

The healer searched Yrene's eyes for the answer she wanted. The answer Yrene could not give. So Yrene just squeezed the woman's hand again and stared into the darkness.

\emph{You must enter where you fear to tread.}

Yrene knew what she had to do. And wished she didn't.

\includegraphics[width=1.12in,height=0.24in]{images/seperator}

"Well? Has Yrene healed you yet?"

Seated at the high table in the khagan's great hall, Chaol turned to where Princess Hasar sat several seats down. A cooling breeze that smelled of oncoming rain flowed through the open windows to rustle the white deathbanners hanging from their upper frames.

Kashin and Sartaq glanced their way---the latter giving his sister a disapproving frown.

"Talented as Yrene may be," Chaol said carefully, aware that many listened even without acknowledging them, "we are only in the initial stages of what will likely be a long process. She left this afternoon to do some research at the Torre library."

Hasar's lips curled into a poisoned smile. "How fortunate for you, that we shall have the pleasure of your company for a while yet." As if he'd willingly stay here for a moment longer.

But Nesryn answered, still glowing from hours again spent with her family that afternoon, "Any chance for our two lands to build bonds is a fortunate one."

"Indeed," was all Hasar said, and went back to picking at the chilled tomatoand-okra dish on her plate. Her lover was nowhere to be seen---but neither was Yrene. The healer's fear earlier  he'd been able to almost taste it in the air. But sheer will had steadied her---will and temper, Chaol supposed. He wondered which would win out in the end.

Indeed, some small part of him hoped Yrene would stay away, if only to avoid what she so heavily implied they'd also be doing: \emph{talking}. Discussing things. Himself.

He'd make it clear to her tomorrow that he could heal just fine without it.

For long minutes, Chaol remained in silence, marking those at the table, the servants flitting by. The guards at the windows and archways.

The minced lamb turned leaden in his stomach at the sight of their uniforms, at how they stood so tall and proud. How many meals had he himself been positioned by the doors, or out in the courtyard, monitoring his king? How many times had he laid into his men for slouching, for chattering amongst themselves, and reassigned them to lesser watches?

One of the khagan's guards noticed his stare and gave a curt nod.

Chaol looked away quickly, his palms clammy. But he forced himself to keep observing the faces around him, what they wore and how they moved and smiled.

No sign---none---of any wicked force, whether dispatched from Morath or elsewhere. No sign beyond those white banners to honor their fallen princess.

Aelin had claimed the Valg had a reek to them, and he'd seen their blood run black from mortal veins more times than he cared to count, but short of demanding everyone in this hall cut open their hands 

It actually wasn't a bad idea---if he could get an audience with the khagan to convince him to order it. To mark whoever fled, or made excuses.

An audience with the khagan to convince him of the danger, and perhaps make \emph{some} progress with this alliance. So that the princes and princesses sitting around him might never wear a Valg collar. Their loved ones never know what it was to look into their faces and see nothing but ancient cruelty smirking back.

Chaol took a steadying breath and leaned forward, to where the khagan dined a few seats down, immersed in conversation with a vizier and Princess Duva.

The khagan's now-youngest seemed to watch more than participate, and though her pretty face was softened with a sweet smile, her eyes missed nothing. It was only when the vizier paused for a sip of wine and Duva turned toward her quiet husband on her left that Chaol cleared his throat and said to the khagan, "I would thank you again, Great Khagan, for offering the services of your healers."

The khagan slid weary, hard eyes toward him. "They are no more my healers than they are yours, Lord Westfall." He returned to the vizier, who frowned at Chaol for interrupting.

But Chaol said, "I was hoping to perhaps be granted the honor of a meeting with you in private."

Nesryn dug her elbow into his in warning as silence rippled down the table. Chaol refused to take his stare from anywhere but the man before him.

The khagan only said, "You may discuss such things with my Chief Vizier, who maintains my daily schedule." A jerk of the chin toward a shrewd-eyed man monitoring from down the table. One glance at the Chief Vizier's thin smile told Chaol the meeting wasn't going to happen. "My focus remains on assisting my wife through her mourning." The gleam of sorrow in the khagan's eyes wasn't feigned. Indeed, there was no sign of the khagan's wife at the table, not even a place left out for her.

Distant thunder grumbled in the thick silence that followed. Not the time or the place to insist. A man grieving for a lost child 
He'd be a fool to push. And coarse beyond measure.

Chaol dipped his chin. "Forgive me for intruding in this difficult time." He ignored the smirk twisting Arghun's face while the prince observed from his father's side. Duva, at least, offered him a sympathetic smile-wince, as if to say, \emph{You are not the first to be shut down}. \emph{Give him time}.

Chaol gave the princess a shallow nod before returning to his own plate. If the khagan was set on ignoring him, grief or no  perhaps there were other avenues to convey information.

Other ways to gain support.

He glanced to Nesryn. She'd informed him when she'd returned before dinner that she'd had no luck seeking out Sartaq this morning. And now, with the prince seated across from them, sipping from his wine, Chaol found himself casually asking, "I heard that your legendary ruk, Kadara, is here, Prince."

"Ghastly beast," Hasar muttered halfheartedly into her okra, earning a half smile from Sartaq.

"Hasar is still sore that Kadara tried to eat her when they first met," Sartaq confided.

Hasar rolled her eyes, though a glimmer of amusement shone there.

Kashin supplied from a few seats down, "You could hear her screeching from the harbor."

To Chaol's surprise, Nesryn asked, "The princess or the ruk?"

Sartaq laughed, a startled, bright sound, his cool eyes lighting. Hasar only gave Nesryn a warning look before turning to the vizier beside her.

Kashin grinned at Nesryn and whispered, "Both."

A chuckle escaped Chaol's throat, though he reined it in at Hasar's glare. Nesryn smiled, inclining her head in good-willed apology to the princess.

Yet Sartaq watched them closely over the rim of his golden goblet. Chaol asked, "Are you able to fly Kadara much while you're here?"

Sartaq didn't miss a beat as he nodded. "As often as I can, usually near dawn. I was in the skies right after breakfast today, and returned just in time for dinner, thankfully."

Hasar muttered to Nesryn without breaking from the vizier commanding her attention, "He's never missed a meal in his life."

Kashin barked a laugh that had even the khagan down the table glancing their way, Arghun scowling with disapproval. When had the royals last laughed since their sister's passing? From the khagan's tight face, perhaps a while.

But Sartaq tossed his long braid over a shoulder before patting the flat, firm stomach beneath his fine clothes. "Why do you think I come home so often, sister, if not for the good food?"

"To plot and scheme?" Hasar asked sweetly.

Sartaq's smile turned subdued. "If only I had time for such things."

A shadow seemed to pass over Sartaq's face---and Chaol marked where the prince's gaze drifted. The white banners still streamed from the windows set high in the walls of the hall, now caught in what was surely the heralding wind of a thunderstorm. A man who perhaps wished he'd possessed extra time for more vital parts of his life.

Nesryn asked a touch softly, "You fly every day, then, Prince?"

Sartaq dragged his stare from his youngest sister's death-banners to assess Nesryn. More warrior than courtier, yet he nodded---in answer to an unspoken request. "I do, Captain."

When Sartaq turned to respond to a question from Duva, Chaol exchanged a glance with Nesryn---all he needed to convey his order.

\emph{Be in the aerie at dawn. Find out where he stands in this war.}

